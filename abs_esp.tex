%\begin{center}
%\large \bf \runtitulo
%\end{center}
%\vspace{1cm}
\chapter*{\runtitulo}

\noindent
La clasificación multi-etiquetas es un paradigma de aprendizaje supervisado que generaliza las técnicas clásicas de clasificación para abordar problemas en donde cada instancia de una colección se encuentra asociada a múltiples etiquetas. La mayor parte de los trabajos de investigación han sido realizados en contextos de aprendizaje por \textit{batch}. Los ambientes de flujo continuo de datos  (o \textit{streaming}) presentan nuevos desafíos a esta área debido a las limitaciones de tiempo de respuesta y almacenamiento que acarrean. A esto se agrega la naturaleza evolutiva de este tipo de escenarios, que obligan a los algoritmos a adaptarse a cambios de concepto. En la presente investigación se aplican algoritmos de clasificación multi-etiquetas a colecciones estructuradas y no estructuradas. Los experimentos se llevarán a cabo en ambientes simulados de \textit{streaming} de datos para conocer el impacto que produce este contexto sobre los resultados de la clasificación y acoplar el modelo a escenarios del mundo real. A su vez, se partirá de estas colecciones de datos para generar instancias sintéticas y así producir flujos potencialmente infinitos. Por último, se abordarán estrategias de ensambles de algoritmos en búsqueda de una mejora en la calidad de la tarea de predicción de objetos no observados por el modelo. De esta manera, se proveerá a la comunidad de nuevos estudios experimentales sobre algoritmos y colecciones ya conocidos del área de clasificación multi-etiquetas, de manera tal de extender el conocimiento sobre su rendimiento bajo escenarios evolutivos y de naturaleza variable.

\bigskip

\noindent\textbf{Palabras claves:} clasificación, multi-etiquetas, \textit{streaming}, algoritmos, flujos.
