%\begin{center}
%\large \bf \runtitulo
%\end{center}
%\vspace{1cm}
\chapter*{\runtitulo}

\noindent La clasificación multi-etiquetas es un paradigma de aprendizaje
supervisado que generaliza las técnicas clásicas de clasificación para abordar
problemas en donde cada instancia de una colección se encuentra asociada a
múltiples etiquetas. La mayor parte de los trabajos de investigación han sido
realizados en contextos de aprendizaje por \textit{batch}. Los ambientes de
flujo continuo de datos  (o \textit{streaming}) presentan nuevos desafíos a esta
área debido a las limitaciones de tiempo de respuesta y almacenamiento que
acarrean. En la presente investigación se aplican algoritmos de clasificación
multi-etiquetas a colecciones estructuradas y no estructuradas. Los experimentos
se llevaron a cabo en ambientes simulados de \textit{streaming} de datos para
conocer el impacto que produce este contexto sobre los resultados de la
clasificación y acoplar el modelo a escenarios del mundo real. A su vez, se
partió de estas colecciones de datos para generar instancias sintéticas y así
producir flujos potencialmente infinitos. A este fin se presenta un método de
generación de instancias sintéticas que busca replicar fenómenos particulares de
colecciones multi-etiquetadas. Por último, se diseña una estrategia de ensambles
de algoritmos, \acrshort{efmp}, en búsqueda de una mejora en la calidad de la
tarea de predicción de objetos no observados por el modelo. De esta manera, se
provee a la comunidad de nuevos estudios experimentales sobre algoritmos y
colecciones ya conocidos del área de clasificación multi-etiquetas, de manera
tal de extender el conocimiento sobre su rendimiento bajo escenarios evolutivos
y de naturaleza variable.

\bigskip

\noindent\textbf{Palabras claves:} clasificación, multi-etiquetas, \textit{streaming}, algoritmos, flujos.
