\chapter{Metodología}

Conforme a los objetivos planteados en la sección~\ref{intro_objetivos} se
presentan las colecciones y algoritmos que serán evaluados junto con el
escenario de flujos continuos diseñado a este fin.

Se seleccionan tres colecciones de datos multi-etiquetados que han sido puntos
de referencia en otros trabajos de investigación. Cada uno de ellos es
transformado en un flujo continuo de datos que cumple con las características
presentadas en la sección~\ref{stream_caracteristicas}. A su vez, se generan
instancias sintéticas que sean fieles a las cualidades subyacentes de los datos
originales.  Para ello, se ha diseñado un algoritmo basado en la implementación
de \citeauthor{read_multi-label_2008} \cite{read_multi-label_2008} pero que hace
uso de la matriz de probabilidades condicionales entre pares de etiqueta para
respetar sus interdependencias.

La etapa de entrenamiento se lleva a cabo con algoritmos de clasificación
multi-etiquetas que han sido adaptados a ambientes de flujos continuos para
hacer frente a las cualidades incrementales inherentes a este contexto. Se
seleccionan algoritmos de la familia de \comillas{Transformación del problema} y
\comillas{Adaptación del algoritmo}, junto con soluciones de ensamble, a fines
explorativos y para extender el conocimiento sobre sus fortalezas y debilidades.
Al mismo tiempo, se diseña un algoritmo de ensambles del tipo \textit{bagging}
\todo{es bagging?} que aplica mayoría de votos ponderada, basada en??
\todo{citar fuente}, y se implementa en una librería de código del lenguaje
Python \todo{link a web de python?}.

Para la etapa de evaluación se sigue la estrategia
\comillas{\textit{Prequential}}, descripta en la
sección~\ref{stream_evaluacion}, se aplican métricas basadas en etiquetas y
ejemplos, se mide la eficiencia de los modelos en términos de velocidad y
espacio de almacenamiento y se analizan los resultados obtenidos.

El marco metodológico de este proyecto se ajusta a los procedimientos efectuados
por los investigadores de la literatura de referencia. Con ello se busca
expandir el conocimiento empírico de los algoritmos al mismo tiempo que
proporcionar nuevos estudios que puedan ser contrastables con los ya existentes.
De la literatura relacionada se destaca el trabajo de
\citeauthor{read_scalable_2012} \cite{read_scalable_2012}, quienes analizan
algoritmos multi-etiquetados con flujos reales y sintéticos, pero a diferencia
de este trabajo, generan instancias sintéticas para colecciones nuevas y sin
basarse en colecciones reales específicas. La implementación del generador que
usaron en sus experimentos se encuentra disponible al público y ha sido el punto
de partida para desarrollar la técnica propuesta. \todo{mencionar parametros z y
	u aquí?}

\todo[inline]{mencionar otros trabajos: souza, gooweml, etc}

\section{Técnicas Propuestas}

En esta sección se describen dos técnicas implementadas para este trabajo: un
generador de instancias sintéticas para flujos continuos de datos y el diseño de
una solución de ensambles para realizar clasificaciones.

\subsection{Generación de Flujos Sintéticos}

El generador presentado es un algoritmo que emplea técnicas probabilísticas para
hallar dependencias dependencias entre etiquetas y reproducirlas en las nuevas
instancias. La existencia de interdependencias entre etiquetas ha sido explorada
reiteradas veces en la literatura \cite{tsoumakas_multi-label_2007,
	read_multi-label_2008} y se ha demostrado que existen dependencias condicionales
e incondicionales, esto es, etiquetas que dependen entre sí dado uno o más
atributos de una instancia (dependencia condicional), y etiquetas cuya
dependencia existe para todo el conjunto de instancias (dependencia
incondicional). Estas dos cualidades son directamente extraídas de las colección
real y a partir de ellas se generan los atributos y etiquetas de la instancia
sintética.

La dependencia incondicional parte de la idea de que hay etiquetas que se
activan en conjunto con frecuencia y otras que son mutuamente excluyentes. Véase
el caso de las etiquetas \comillas{\texttt{Ficción}} y \comillas{\texttt{No
		Ficción}}, por ejemplo, que son excluyentes en el dominio de géneros literarios.
Para capturar esta relación se acude al concepto de probabilidad a priori y
probabilidad condicional de etiquetas. La probabilidad a priori de una etiqueta
es obtenida a partir de observar su frecuencia relativa en la colección
normalizada por la cardinalidad de etiquetas. \todo{Chequear fórmula con
	director!} La frecuencia relativa se formula de la manera tradicional:

\begin{equation}
	FrecRelE_{j} = \frac{1}{m} \sum_{i=1}^{m} y_{i,j}
\end{equation}

La normalización toma en cuenta el valor de cardinalidad de etiquetas del
conjunto de datos, esto bajo la recomendación de los autores del trabajo de
referencia \cite{read_scalable_2012}.

\begin{equation}
	NormCardE = \frac{1}{CardE} \sum_{j=1}^{q} FreqRelE_{j}
\end{equation}

Luego, la probabilidad a priori de la etiqueta $j$ se expresa de la forma:

\begin{equation}
	P(E_{j}) =\min{(1, \frac{FreqRelE_{j}}{NormCardE})}
\end{equation}

El resultado es un vector de probabilidades a priori $[P(E_{1}),
			P(E_{2}), \dots, P(E_{q})]$ que expresa la probabilidad independiente de cada
etiqueta.

A partir de $P(E_{j})$ se puede calcular la matriz condicional $\theta$ sobre
los pares de etiquetas, esto es, $\theta_{j,k} = P(Y_{j} = 1 \mid Y_{k} = 1)$,
donde $L \geq j > k \leq 1$. \todo{L es menor o igual a j en la bibliografia, es
	un error?} Con el vector de probabilidades a priori y extrayendo las
co-ocurrencias de cada par de etiquetas en toda la colección, es posible obtener
cada valor de la matriz $\theta$, aplicando la probabilidad condicional:

\begin{equation}
	P(Y_{j} = 1 \mid Y_{k} = 1) = \frac{P(Y_{k} = 1 \cap Y_{j} = 1)}{P(Y_{k})}
\end{equation}

Luego, la dependencia entre etiquetas es modelada como la distribución conjunta:

\begin{equation}
	\label{eq:syn_joint}
	p_{\theta}(y) = P(y_{1}) \prod_{j=2}^q P(y_{j} \mid y_{j-1})
\end{equation}

Posteriormente, se realiza la generación del conjunto de etiquetas para la
instancia sintética. El algoritmo~\ref{alg:generar_etiquetas} muestra las
instrucciones ejecutadas para concretar esta tarea. Cabe aclarar que $sample()$
retorna un índice de etiqueta de acuerdo a una función de masa de probabilidad
basada en las probabilidades a priori, y $random()$ produce un número aleatorio
de distribución uniforme.

\begin{center}
	\begin{algorithm}[H]
		\label{alg:generar_etiquetas}
		\SetAlgoLined
		\DontPrintSemicolon
		\KwIn{
			$q$:  Número de etiquetas de la colección,
			$p$: vector de probabilidades a priori,
			$p_{\theta}(y)$: función definida en fórmula~\ref{eq:syn_joint}
		}
		\KwOut{$y$: las etiquetas generadas.}
		$y \gets \emptyset_{q}$\;
		$j \gets sample(p)$\;
		$y_{j} \gets 1$\;
		$i \gets 0$ \;
		\While{$i < q$}{
			\lIf{$i = j$}{$\Continue$}
			$y^{\prime} \gets y$\;
			$y^{\prime}_{i} \gets 1$\;
			\lIf{$p_{\theta}(y^{\prime}) > random()$}{$y \gets y^{\prime}$ }
			$i \gets i+1$\;
		}
		\caption{Algoritmo de generación del conjunto de etiquetas para una instancia
			sintética}
	\end{algorithm}
\end{center}

Una vez generado el conjunto de etiquetas resta generar los valores de atributos
para la instancia. Para ello se retoma el concepto ya mencionado de
\comillas{Dependencia Condicional}, para conocer en qué medida la presencia de
un atributo activa una o más etiquetas en la instancia, o expresado en términos
formales, hallar el término $P(y|x)$ tal que:

\begin{equation}
	P(y|x) = P(x|y)P(y)
\end{equation}

Como calcular la probabilidad conjunta es altamente complejo se define una
función de mapeo $\zeta[a] \mapsto y_{a}$, donde $y_{a}$ es la combinación de
etiquetas más probable para el atributo $a$.  La función $\theta$ se obtiene a
través de muestreos sucesivos del generador de etiquetas, y guardando las $A$
combinaciones más frecuentes, siendo el número total de atributos. Al mismo
tiempo, el vector $x$ candidato es obtenido usando un generador binario tal como
los descriptos en la sección~\ref{stream_syn}. El algoritmo~\ref{alg:generar_atributos} muestra un pseudocódigo de cómo se completa el
proceso. Notar que el generador binario $g$ produce dos vectores de atributos
candidatos, uno por cada clase, luego si la combinación de etiquetas para el
atributo $a$ es un subconjunto de las etiquetas generadas se toma el valor del
vector de atributos positivos. Caso contrario, se toma del vector de negativos.

Finalmente, la instancia sintética se forma a partir de la salida de ambos
algoritmos, siendo de la forma $(x, y)$. Este proceso será repetido para cada
instancia que se solicite al generador a fin de generar el flujo sintético para
la colección dada. El objetivo es obtener colecciones sintéticas que se asemejen
a datos del mundo real, por lo tanto, la evaluación de los resultados se hará
mediante un análisis de sus cualidades en relación a fenómenos hallados en datos
reales (ver sección~\ref{mll_fenomenos}), y se contrastan los datos generados en
este marco contra los producidos en el trabajo de referencia.

\begin{center}
	\begin{algorithm}[H]
		\label{alg:generar_atributos}
		\SetAlgoLined
		\DontPrintSemicolon
		\KwIn{
			$A$:  Número de atributos de la colección,
			$g$: Generador de atributos,
			$\zeta$: Función de mapeo.
		}
		\KwOut{$x$: El vector de atributos generado.}
		$x \gets \emptyset_{A}$\;
		$positivos \gets g(1)$ \;
		$negativos \gets g(0)$ \;
		$i \gets 0$ \;
		\While{$i < A$}{
			\uIf{$\exists q : \zeta[a] \subseteq y_{q}$}{
				$x_{i} \gets positivos_{i}$ \;
			}
			\Else{
				$x_{i} \gets negativos_{i}$ \;
			}
			$i \gets i+1$ \;
		}
		\caption{Algoritmo de generación del conjunto de etiquetas para una instancia
			sintética}
	\end{algorithm}
\end{center}

\subsection{Algoritmo de Ensamble}
