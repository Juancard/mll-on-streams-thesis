\chapter{Conclusiones}

En este trabajo se estudia la tarea de clasificación multi-etiquetas en
ambientes de flujos continuos de datos. Por un lado, se aplican métodos de
generación de instancias sintéticas para obtener \textit{streamings} de
múltiples etiquetas que sean representativos de escenarios del mundo real. A los
parámetros de configuración ya existentes, se le agrega la posibilidad de
generar instancias nuevas para una colección dada, teniendo en cuenta la matriz
de co-ocurrencia de etiquetas. A su vez, se realiza una comparativa de
rendimientos entre algoritmos de \acrshort{mll} bajo un conjunto de métricas de
evaluación seleccionadas y siguiendo las metodologías y directivas utilizadas
por otros autores del campo de estudio, a fin de conocer la capacidad predictiva
de los modelos sobre distintos escenarios de \textit{streamings}. Este análisis
derivó en el diseño y desarrollo de una nueva solución de ensambles del tipo de
votación por mayoría, llamada \acrshort{efmp}, que toma como clasificadores base
a un conjunto fijo de algoritmos clásicos del área y clasifica una instancia de
acuerdo a su rendimiento previo, penalizando a aquellos clasificadores que no
han acertado la clase correcta de una etiqueta.

Con respecto a la generación de flujos continuos sintéticos, se puede aseverar
que los métodos disponibles cuentan con parámetros de configuración que permiten
generar datos cercanos a los del mundo real, por lo menos con respecto a los
fenómenos aquí analizados, que son los de sesgo de etiquetas, distribución
de etiquetas, relación entre etiquetas y espacio de atributos.  Además, la
inclusión de un parámetro más para indicar la matriz de co-ocurrencia de
etiquetas ha dado resultados que vale la pena mencionar. En primer lugar, el uso
de la matriz derivada de la colección 20ng ha contribuido a generar un
\textit{stream} sintético con mayor cercanía al del método de MOA para todos los
fenómenos estudiados. Incluso en el estudio del sesgo de etiquetas para Enron se
observa una curva de sesgo más próxima a la de la colección original. De
cualquier manera, para esta última colección y para Mediamill no es posible
observar una mejoría significativa en cuanto a la distribución de etiquetas con
respecto a MOA y, por lo tanto, no es posible determinar con certeza que un
método sea mejor que el otro para simular estos datos. En cuanto al análisis de
la relación entre etiquetas se observa que el uso de la matriz de co-ocurrencias
ha contribuido a aproximar con mayor cercanía la combinación entre pares de
etiquetas de los datos originales para las tres colecciones estudiadas. Además,
si bien la frecuencia de co-ocurrencia es visiblemente menor a la de las
colecciones originales, es de notar que la cercanía es mayor en comparación al
método ya conocido de \textit{MOA}\@. Por último, es interesante analizar si la
mejoría en la generación de instancias sintéticas para 20ng se extiende a otras
colecciones de tipo A, es decir, colecciones donde la mayoría de las instancias
tienen una única etiqueta. En lo que respecta a este estudio es posible observar
que la similitud de los flujos continuos sintéticos ha sido mayor para las
colecciones de tipo A (20ng) que para las de tipo B (Enron y Mediamill). No
obstante, es necesario realizar más estudios para confirmar este patrón.  En
definitiva, se puede concluir que los métodos existentes contribuyen a obtener
datos que son lo suficientemente representativos de datos del mundo real como
para conducir estudios y evaluaciones de algoritmos de \acrshort{mll} en
ambientes de \textit{streamings}.

En cuanto a la tarea de clasificación, se llevaron a cabo evaluaciones
multi-etiquetas bajo una serie de métricas comúnmente utilizadas en tareas de
\acrshort{mll} y sobre colecciones de datos bien conocidas en la literatura.
Para ello, se seleccionaron algoritmos clásicos del área, soluciones de
ensambles y dos versiones del algoritmo \acrshort{efmp}, que han permitido
examinar las fortalezas y debilidades de cada modelo de clasificación. Los
resultados muestran que el algoritmo propuesto logra heredar el rendimiento
predictivo de métodos \textit{baselines} como \acrshort{mlht}, \acrshort{br} y
\acrshort{cc} y, a partir de ello, obtener valores en las métricas de evaluación
que son competitivos con respecto a los métodos existentes. En la comparativa
con otras soluciones de ensambles los resultados son favorecedores para los
algoritmos presentados. \acrshort{efmp} y su variante producen mejores valores
para buena parte de las métricas basadas en ejemplos y etiquetas
(\textit{exact-match}, \textit{accuracy}, precisión, \textit{recall} y
\textit{f1}) e incluso mejoran la eficiencia de todos a excepción de
\acrshort{ebr} para Mediamill. También en varios casos \acrshort{efmp} superó a
los \textit{baselines}. Bajo la métrica de \textit{f1} basada en ejemplos, esta
propuesta obtiene los valores más altos para 20ng y Enron y es el segundo mejor
para Mediamill; bajo la métrica de \textit{f1} macro, \acrshort{efmp} es la
mejor para 20ng y \acrshort{efmp2} para Mediamill; y bajo la métrica de
\textit{f1} micro, \acrshort{efmp} supera a todas las demás propuesta para 20ng
y Enron.

También en este trabajo se conduce una comparativa entre los resultados aquí
presentados y los de otros autores en sus publicaciones. En este sentido, los
resultados son mixtos. En algunos escenarios, como por ejemplo Enron, nuestra
propuesta logra superar e incluso duplicar el rendimiento de otros algoritmos,
tales como iSOUP-RT o ML-SAM-kNN\@. En otros escenarios, como Mediamill o 20ng,
los resultados son menos favorecedores. De cualquier modo, las soluciones
propuestas son competitivas en todos los casos.

Una cuestión que vale la pena mencionar con respecto a soluciones de ensambles
como las aquí propuestas tiene que ver con el rendimiento bajo métricas de
eficiencia como el tiempo de ejecución y el tamaño del modelo. Se puede
observar, a partir de los resultados presentados, que el uso de métodos de
computabilidad más compleja tienden a lanzar mejores resultados, pero acarrean
un mayor costo en cuanto a recursos. Esto debe ser considerado al momento de
diseñar aplicaciones del mundo real donde la disponibilidad de recursos puede
ser muy limitada y podría ser adecuado optar por un modelo más simple.

En conclusión, la clasificación multi-etiquetas representa un gran desafío en sí
mismo, y que se incrementa en un contexto de \textit{streamings} de datos. A
esto se suma la baja disponibilidad de colecciones públicas que tengan las
dimensiones necesarias para conducir experimentos acordes a escenarios del mundo
real. Bajo estas restricciones, todas las mejoras tanto en la generación de
instancias sintéticas como en el rendimiento de algoritmos de clasificación son
un aporte muy valioso al campo, en vías de incrementar la capacidad de
predicción de los modelos.

\section{Trabajos Futuros}

Habiendo alcanzado la etapa final del proyecto, se plantean una serie de
interrogantes que podrían derivar en lineas de trabajo futuras:

\begin{itemize}

	\item Las evaluaciones de algoritmos \acrshort{mll} aquí realizadas pueden
	      ser complementadas con estudios que saquen provecho de datos a gran
	      escala, esto es, usando flujos continuos sintéticos y multi-etiquetados
	      generados con los métodos aquí analizados. Esto permitiría llevar a cabo
	      investigaciones en un contexto de \textit{streaming} más cercano a los
	      del mundo real.

	\item Una característica preponderante en los ambientes de
	      \textit{streamings} es la existencia de cambios de concepto en los
	      datos, lo que significa que los datos pueden variar de distribución a
	      lo largo del tiempo y de manera impredecible. Es por ello que sería de
	      interés generar escenarios de flujos continuos que produzcan cambios
	      de concepto y realizar evaluaciones de los algoritmos. A su vez, se
	      podrían añadir a las pruebas otros algoritmos ya diseñados a este fin
	      para enriquecer el estudio.

	\item Con respecto a las variantes de \acrshort{efmp} aquí presentadas, es
	      de interés estudiar su rendimiento en ambientes que presenten cambios
	      de concepto e implementar técnicas que mejoren su performance en dicho
	      contexto. A este fin, una posible linea de investigación es añadir la
	      posibilidad de que el algoritmo de ensambles reemplace de manera
	      dinámica a aquellos clasificadores base menos eficaces durante la fase
	      de entrenamiento, de manera tal que al producirse un cambio de
	      concepto, el modelo pueda adaptarse en una ventana de tiempo deseable.
	      Esta técnica es similar a la aplicada por los modelos de
	      \acrshort{dwm}.

	\item Se ha mencionado anteriormente que el uso de la matriz de
	      co-ocurrencias en la generación de datos sintéticos puede ser más
	      conveniente cuando se trata de aproximar colecciones de tipo A como
	      20ng. Para confirmar esta suposición, un enfoque posible es conducir
	      experimentos con más colecciones de este tipo y comparar los resultados
	      contra otros flujos sintéticos generados con colecciones de tipo B.

	\item El análisis de las evaluaciones realizadas por las variantes de
	      \acrshort{efmp} arrojó resultados sub-óptimos para las métricas de
	      eficiencia, tal como sucedió con otros modelos de ensambles utilizados
	      en este trabajo. Por lo tanto, es de interés buscar caminos que
	      conduzcan a mejorar la velocidad de ejecución en la fase de
	      entrenamiento. En principio, sería viable adaptar la implementación de
	      los algoritmos para que sean ejecutables bajo una arquitectura
	      distribuida y paralela, donde cada clasificador base pueda realizar sus
	      predicciones de manera independiente, es decir, sin esperar que finalice
	      el clasificador anterior en forma secuencial. De esta forma, el tiempo
	      de ejecución en la predicción de los clasificadores base sería igual al
	      tiempo de aquel clasificador más lento, lo que podría derivar en una
	      mejora significativa en la eficiencia.

\end{itemize}
